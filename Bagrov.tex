\documentclass[
11pt,%
tightenlines,%
twoside,%
onecolumn,%
nofloats,%
nobibnotes,%
nofootinbib,%
superscriptaddress,%
noshowpacs,%
centertags]%
{revtex4}
\usepackage{ljm}
\usepackage{listings}
\usepackage{amsmath}

\lstset{
language=C++,
basewidth=0.5em,
xleftmargin=45pt,
xrightmargin=45pt,
basicstyle=\small\ttfamily,
keywordstyle=\bfseries\underbar,
numbers=left,
numberstyle=\tiny,
stepnumber=1,
numbersep=10pt,
showspaces=false,
showstringspaces=false,
showtabs=false,
frame=trBL,
tabsize=2,
captionpos=t,
breaklines=true,
breakatwhitespace=false,
escapeinside={\%*}{*)}
}

\begin{document}

\titlerunning{solving nonlinear equations}
\authorrunning{Bagrov, Rybakov}

\title{Selection of a Method for Solving Nonlinear Equations in Shallow-Water Icing Model Implementation}

\author{\firstname{A.~D.}~\surname{Bagrov}}
\email[E-mail: ]{andrey.bagrov@yandex.ru}
\affiliation{Joint Supercomputer Center of the Russian Academy of Sciences -- branch of Scientific Research Institute of System Analysis of the Russian Academy of Sciences, Leninsky prospect 32a, Moscow, 119334, Russia}

\author{\firstname{A.~A.}~\surname{Rybakov}}
\email[E-mail: ]{rybakov.aax@gmail.com}
\affiliation{Joint Supercomputer Center of the Russian Academy of Sciences -- branch of Scientific Research Institute of System Analysis of the Russian Academy of Sciences, Leninsky prospect 32a, Moscow, 119334, Russia}

\firstcollaboration{(Submitted by TODO)} % Add if you know submitter.
%\lastcollaboration{ }

\received{TODO}

\begin{abstract}
Моделирование ледяных наростов на профилях летательных аппаратов в процессе их полета в среде, содержащей переохлажденные капли воды, является крайней важной задачей для безопасности полетов, так как образуемые ледяные наросты существенным образом влияют на летные характеристики.
В одной из моделей решения поставленной задачи -- shallow-water icing model (SWIM) -- центральную роль в численном моделировании занимает задача решения нелинейных уравнений с одной переменной.
Так как данная задача занимает подавляюще большую часть всех расчетов, то вопрос выбора оптимального метода решения нелинейных уравнений и оптимизации данных методов встает особенно остро.
В данной статье описан анализ использования различных методов решения нелинейных уравнений в реализации решателя SWIM с учетом особенностей решаемых уравнений и продемонстрировано существенное ускорение расчетных кодов при выполнении вычислений на суперкомпьютерах МСЦ РАН.
\end{abstract}

\subclass{65H05, 65Y20} % Enter 2010 Mathematics Subject Classification.

\keywords{nonlinear equations, shallow-water icing model, метод бисекций, метод ньютона, метод брента, TODO}

\maketitle

\section{Introduction}

В настоящее время известно множество расчетных кодов, использующихся для численного моделирования обледенения поверхности обтекаемого тела.
Одними из наиболее популярных пакетов для решения данной задачи являются Lewice \cite{Wright} и ONERA.
В данной статье мы не будем касаться особенностей данных расчетных пакетов и различий между ними, а рассмотрим только реализацию решателя SWIM, подробно описанного в \cite{Bourgault}.
При выполнении компьютерного моделировании процесса обледенения поверхности, обтекаемой свободным потоком, в shallow-water icing model выполняется одновременный расчет нарастания льда и течения пленки жидкости по поверхности обтекаемого тела. При этом учитывается выпадение влаги на поверхность тела, испарение влаги или сублимация льда с поверхности тела, течение жидкой пленки по поверхности с частичным ее замерзанием, а также перетекание потоков тепла между телом и поверхностью и между поверхностью и окружающим воздухом.

Численные расчеты выполняются на поверхностной расчетной сетке, состоящей из отдельных ячеек, при этом в каждой ячейке поверхности должен выполняться закон сохранения массы, записываемый в следующем виде:

\begin{equation}
\rho_w \left[ \frac{\partial h_f}{\partial t} + \operatorname{div}(\overline{u} h_f) \right] = U_{\infty} LWC \beta - \dot m_{evap} - \dot m_{ice}
\end{equation}

В данной формуле $\rho_w$ -- плотность воды, $h_f$ -- высота водяной пленки на поверхности, $\overline{u}$ -- скорость течения водяной пленки, $U_{\infty}$ -- скорость набегания свободного потока, $LWC$ -- содержание влаги в набегающем потоке, $\dot m_{evap}$ -- удельная скорость испарения или сублимации с поверхности обтекаемого тела, $\dot m_{ice}$ -- удельная скорость нарастания массы льда.

Кроме того, в каждой ячейке выполняется закон сохранения энергии:

\begin{equation}
\begin{aligned}
& \rho_w \left[ \frac{\partial h_f C_w \tilde{T}}{\partial t} + \operatorname{div}(\overline{u} h_f C_w \tilde{T}) \right] = \left[ C_w \tilde{T}_{d,\infty} + \frac{||\overline{u}_d||^2}{2} \right] \times U_{\infty} LWC \beta
\\
& - \frac{1}{2}(L_{evap} + L_{subl}) \dot m_{evap} + (L_{fusion} - C_{ice} \tilde{T}) \dot m_{ice} + \sigma (T_{\infty}^4 - T^4) + \dot Q_h + \dot Q_{cond}
\end{aligned}
\end{equation}

В данной формуле $C_w$ -- удельная теплоемкость воды, $\tilde{T}$ -- температура ячейки в градусах Цельсия, $\tilde{T}_{d,\infty}$ -- температура выпадающей влаги в градусах Цельсия, $\overline{u}_d$ -- скорость выпадающих на поверхность капель, $L_{evap}$ - скрытая теплота испарения воды, $L_{subl}$ -- скрытая теплота сублимации льда, $L_{fusion}$ -- скрытая теплота плавления льда, $C_{ice}$ -- теплоемкость льда, $\sigma$ -- постоянная Больцмана, $T_{\infty}$ -- температура свободного потока в кельвинах, $T$ - температура ячейки в кельвинах, $\dot Q_h$ -- удельная величина потока тепла, получаемого из воздуха, $\dot Q_{cond}$ -- удельная величина потока тепла, поступающего в ячейку из обтекаемого тела.

Для численного решения приведенной системы уравнений необходимо выполнить ее дискретизацию по времени и пространстрву, как это показано в \cite{Beaugendre}.
После этого получим систему из двух разностных уравнений, в которые входят три неизвестные величины: температура поверхности $\tilde{T}$, высота водяной пленки $h_f$ и высота ледяного нароста $h_{ice}$.  

Также при решении системы уравнений должны выполняться условия совместимости, записываемые в виде

\begin{equation}
\begin{cases}
h_f \ge 0\\
\dot m_{ice} \ge 0\\
h_f \tilde{T} \ge 0\\
\dot m_{ice} \tilde{T} \le 0
\end{cases}
\end{equation}

\begin{figure}[h]
\setcaptionmargin{5mm}
\onelinecaptionstrue
\includegraphics[width=0.7\textwidth]{pics/surface.pdf}
\captionstyle{normal}\caption{Пространство решение системы уравнений массового и теплового баланса ячейки.}\label{fig:surface}
\end{figure}

Несмотря на то, что неизвестных в уравнении больше, чем самих уравнений, система все равно решается, так как данные переменные не являются полностью независимыми.
Решение ищется из условия того, что ячейка может находиться в одном из трех состояний.
Первое состояние -- running wet -- достигается когда в ячейке полностью отсутствует лед и течет жидкая пленка.
В этом случае температура не может быть отрицательной.
Второе состояние -- glaze icing -- если в ячейке одновременно присутствует и лед, и вода.
В этом случае температура равна нулю градусам по Цельсию.
И наконец, в третьем случае при отрицательной температуре в ячейке не может находиться вода и присутствует только лед.
Общее пространство решений показано на Fig.~\ref{fig:surface}.

\section{Features of the equations being solved}

\begin{figure}[h]
\setcaptionmargin{5mm}
\onelinecaptionstrue
\includegraphics[width=1.0\textwidth]{pics/ph_graphics_h.pdf}
\captionstyle{normal}\caption{Зависимости физических величин от температуры.}\label{fig:surface}
\end{figure}

\begin{figure}[h]
\setcaptionmargin{5mm}
\onelinecaptionstrue
\includegraphics[width=1.0\textwidth]{pics/dq.pdf}
\captionstyle{normal}\caption{TODO}\label{fig:dq}
\end{figure}

\begin{figure}[h]
\setcaptionmargin{5mm}
\onelinecaptionstrue
\includegraphics[width=1.0\textwidth]{pics/dq_rime_wet.pdf}
\captionstyle{normal}\caption{TODO}\label{fig:dq_rime_wet}
\end{figure}

\section{Methods of solving nonlinear equations}

\subsection{Метод деления пополам (метод бисекции)}
Если известно, что на интервале $(a, b)$ функция $f(x)$ имеет корень, для нахождения корня может быть применен метод деления пополам (метод бисекции). Его сущность заключается в следующем наборе шагов:
\begin{itemize}
\item Задать точность нахождения корня - $\epsilon$.
\item Разделить отрезок $(a, b)$ пополам, получить два отрезка.
\item Выбрать из двух получившихся в первом шаге отрезков отрезок, на концах которого выполняется условие $\ sign(f(a)) != sign(f(b))$.
\item Если длина интервала превышает $\epsilon$, повторить деление отрезка пополам. В противном случае корень находится в середине полученного интервала.
\end{itemize}
	
Для достижения заданной точности $\epsilon$ необходимо провести $\log_2\frac{\epsilon_0}{\epsilon}$  операций, где $\epsilon_0$ – длина начального интервала $(a, b)$, на котором находится корень; $(a, b)$ - требуемая точность. Метод бисекции гарантированно найдет корень, если он существует. Если корней несколько, метод найдет один из корней.

\subsection{Метод хорд}
Метод хорд заключается в представлении функции $f(x)$ на отрезке $(a, b)$ хордой, соединяющей концы интервала $(a,b)$. В результате точка пересечения хорды с $y=0$ является искомым корнем уравнения. Если условие остановы не удовлетворено $a-b<0$, алгоритм повторяется на интервале $(a,c)$. 
Итерационная фомула имеет вид:
\begin{equation}
x_{i+1}=x_{i-1}-\frac{f(x_{i-1})(x_i-x_{i-1})}{f(x_i)-f(x_{i-1})}
\end{equation}

Графическое представления метода метода хорд представлено на Fig.~\ref{fig:chords}.

\begin{figure}[h]
\setcaptionmargin{5mm}
\onelinecaptionstrue
\includegraphics[width=1.0\textwidth]{pics/chords-newton.pdf}
\captionstyle{normal}\caption{Иллюстрация работы метода хорд и метода Ньютона}\label{fig:chords-newton}
\end{figure}

\subsection{Метод Ньютона}
Данный метод отличается от вышеописанных тем, что он требует вычисления как функции $f(x)$, так и производной $f^{'}(x)$ в произвольных точках $x$, принадлежащих интервалу $(a, b)$. Этот метод основан на разложении в ряд Тейлора в окрестности точки $x$:
\begin{equation}
f(x+\delta) \approx f(x)+f^{'}(x)\delta+\frac{f^{''}(x)}{2}\delta^2+ ...
\end{equation}

Для малых $\delta$ справедливо $\delta=-\frac{f(x)}{f^{'}(x)}$.
Таким образом, итерационная фомула имеет вид:
\begin{equation}
x_{n+1}=x_n-\frac{f(x_n)}{f^{'}(x_n)}
\end{equation}

Графическое представления метода метода Ньютона представлено на Fig.~\ref{fig:newton}.
\newpage

\subsection{Метод Брента}
Метод Брента является модифицированным методом Деккера, который, в свою очередь, является объединением метода деления пополам с методом секущих.
Сперва рассмотрим метод Деккера. Для каждой итерации метода рассматривается система
\begin{equation}
s=\begin{cases}
b_k - \frac{b_k-b_{k-1}}{f(b_k)-f(b_{k-1})}f(b_k), \text{ if $f(b_k) \neq f(b_{k-1})$}.\\
m = \frac{a_k+b_k}{2} \text{ otherwise}.\\
\end{cases}
\end{equation}
Подразумевается, что на интервале $(a,b)$ функция $f(x)$ имеет корень. Таким образом, на каждой итерации метода Деккера происходит выбор способа уточнения корня функции - при помощи метода деления пополам или же при помощи метода секущих. Слабая сторона метода Деккера - низкая скорость сходимости при малом $b_k-b_{k-1}$. 

Брент предложил проводить дополнительную проверку $b_k-b_{k-1} > \delta$ для выбора метода уточнения корня, что позволяет повысить скорость сходимости метода. Кроме этого, линейная интерполяция в методе секущих была заменена обратной квадратичной интерполяций, реккурентное соотношение которой можно представить в виде
\begin{equation}
x_{n+1}=\frac{f_{n-1}f_n}{(f_{n-2}-f_{n-1})(f_{n-2}-f_n)}x_{n-2}+ \\\frac{f_{n-2}f_n}{(f_{n-1}-f_{n-2})(f_{n-1}-f_n)}x_{n-1}+\frac{f_{n-2}f_{n-1}}{(f_{n}-f_{n-2})(f_{n}-f_{n-1})}x_{n}.
\end{equation}

В данной формуле $f_n=f(x_n)$.

\section{Numerical experiments}

Для анализа использования различных методов решения нелинейных уравнений были решены порядка 35000 уравнений, аналогичных уравнениям, показанным на Fig.~\ref{fig:dq}.
\newpage

\begin{table}[ht!]
\begin{tabular}{llll}
           & Failed, \% & Avg. func calls & Avg. iterations \\
bisections & 0.038      & 78.8            & 31.3            \\
chords     & 0.038      & 11.2            & 3.7             \\
newton     & 0.038      & 10.1            & 3.3             \\
brenth     & 0.038      & 5.5             & 4.5            
\end{tabular}
\end{table}

Принимая во внимание процент нерешенных уравнений, одинаковый для всех методов, можно сделать вывод, что эти уравнения не имеют решения на исследуемом интервале и это не связано с каким-либо методом. Распределение количества вызова функций и количества итераций представлено на Fig.~\ref{fig:calls} и Fig.~\ref{fig:iterations} соответственно. Анализируя распределения, видно, что наиболее эффективным в смысле количества итераций и количества вызовов функции является метод Брента.

\begin{figure}[h!]
\setcaptionmargin{5mm}
\onelinecaptionstrue
\includegraphics[width=1\textwidth]{pics/stat.pdf}
\captionstyle{normal}\caption{Статистика количеств вызово функции $f(x)$ и итераций решения нелинейного уравнения}\label{fig:stat}
\end{figure}

\section{Conclusion}

TODO

\begin{acknowledgments}
The work has been done at the JSCC RAS as part of the state assignment for the topic 0580-2021-0016.
The supercomputer MVS-10P OP, located at the JSCC RAS, was used during the research.
\end{acknowledgments}

\begin{thebibliography}{99}

\bibitem{Wright}
\refitem{article}
W.~W.~Wright, P.~Struck, T.~Bartkus, G.~Addy, {\it ``Recent Advances in the LIWICE Icing Model''}, SAE Technical Paper (2015).

\bibitem{Bourgault}
\refitem{article}
Y.~Bourgault, H. Beaugendre, W. G. Habashi, {\it ``Development of a Shallow-Water Icing Model in FENSAP-ICE''}, Journal of Aircraft, Vol. 37, No. 4, 640--646 (2000).

\bibitem{Beaugendre}
\refitem{misc}
H.~Beaugendre, {\it ``A PDE-Based Approach to In-Flight Ice Accretion''}, A thesis of the degree of Doctor of Philosophy, Department of Mechanical Engineering, McGill University, Montreal, Qu\'ebec (2003).

\bibitem{Dekker}
\refitem{article}
Brent, R.P.. Finding a zero by means of successive linear interpolation. : London, 1969.

\bibitem{Brent}
\refitem{book}
Brent, R.P.. Algorithms for minimization without derivatives. : Prentice-Hall, 1973.

\bibitem{Press}
\refitem{book}
William H. Press, Saul A. Teukolsky, William T. Vetterling, and Brian P. Flannery. 2007. Numerical Recipes 3rd Edition: The Art of Scientific Computing (3rd. ed.). Cambridge University Press, USA.


\end{thebibliography}

\end{document}
